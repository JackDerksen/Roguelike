\chapter{Notes}
\hypertarget{md_notes}{}\label{md_notes}\index{Notes@{Notes}}
\label{md_notes_autotoc_md0}%
\Hypertarget{md_notes_autotoc_md0}%

\begin{DoxyItemize}
\item Main function should perform the initial setup (like calling \textquotesingle{}screen\+\_\+setup\textquotesingle{}), enter the main game loop where the bulk of the game logic will go, and perform cleanup and exit when the game loop is exited.
\item A way to handle getch input\+: 
\begin{DoxyCode}{0}
\DoxyCodeLine{int\ main(void)\ \{}
\DoxyCodeLine{\ \ \ \ if\ (screen\_setup()\ !=\ 0)\ \{}
\DoxyCodeLine{\ \ \ \ \ \ \ \ return\ 1;\ //\ Exit\ if\ screen\ setup\ fails}
\DoxyCodeLine{\ \ \ \ \}}
\DoxyCodeLine{}
\DoxyCodeLine{\ \ \ \ //\ Other\ setup\ functions...}
\DoxyCodeLine{\ \ \ \ //\ map\_setup();}
\DoxyCodeLine{\ \ \ \ //\ player\_setup();}
\DoxyCodeLine{}
\DoxyCodeLine{\ \ \ \ //\ Main\ game\ loop}
\DoxyCodeLine{\ \ \ \ bool\ game\_running\ =\ true;}
\DoxyCodeLine{\ \ \ \ while\ (game\_running)\ \{}
\DoxyCodeLine{\ \ \ \ \ \ \ \ //\ Handle\ input}
\DoxyCodeLine{\ \ \ \ \ \ \ \ int\ ch\ =\ getch();}
\DoxyCodeLine{}
\DoxyCodeLine{\ \ \ \ \ \ \ \ //\ Process\ the\ input}
\DoxyCodeLine{\ \ \ \ \ \ \ \ switch\ (ch)\ \{}
\DoxyCodeLine{\ \ \ \ \ \ \ \ \ \ \ \ case\ KEY\_UP:}
\DoxyCodeLine{\ \ \ \ \ \ \ \ \ \ \ \ \ \ \ \ //\ move\ player\ up}
\DoxyCodeLine{\ \ \ \ \ \ \ \ \ \ \ \ \ \ \ \ break;}
\DoxyCodeLine{\ \ \ \ \ \ \ \ \ \ \ \ case\ KEY\_DOWN:}
\DoxyCodeLine{\ \ \ \ \ \ \ \ \ \ \ \ \ \ \ \ //\ move\ player\ down}
\DoxyCodeLine{\ \ \ \ \ \ \ \ \ \ \ \ \ \ \ \ break;}
\DoxyCodeLine{\ \ \ \ \ \ \ \ \ \ \ \ //\ ...\ other\ cases\ for\ different\ inputs}
\DoxyCodeLine{\ \ \ \ \ \ \ \ \ \ \ \ case\ 'q':}
\DoxyCodeLine{\ \ \ \ \ \ \ \ \ \ \ \ \ \ \ \ game\_running\ =\ false;\ //\ exit\ the\ game\ loop}
\DoxyCodeLine{\ \ \ \ \ \ \ \ \ \ \ \ \ \ \ \ break;}
\DoxyCodeLine{\ \ \ \ \ \ \ \ \}}
\DoxyCodeLine{}
\DoxyCodeLine{\ \ \ \ \ \ \ \ //\ Update\ game\ state}
\DoxyCodeLine{\ \ \ \ \ \ \ \ //\ ...}
\DoxyCodeLine{}
\DoxyCodeLine{\ \ \ \ \ \ \ \ //\ Render\ to\ screen}
\DoxyCodeLine{\ \ \ \ \ \ \ \ //\ ...}
\DoxyCodeLine{\ \ \ \ \}}
\DoxyCodeLine{}
\DoxyCodeLine{\ \ \ \ //\ Cleanup\ before\ exiting}
\DoxyCodeLine{\ \ \ \ endwin();}
\DoxyCodeLine{\ \ \ \ return\ 0;}
\DoxyCodeLine{\}}

\end{DoxyCode}

\end{DoxyItemize}

In this structure, getch() is called at the start of each iteration of the game loop. It waits for a key press and then processes the input. This way, the game can respond to player actions continuously, and you have the logic in place to exit the game loop (for instance, when \textquotesingle{}q\textquotesingle{} is pressed, as shown in the example).

The previous placement of getch() before the game loop was more of a placeholder to demonstrate waiting for any user input before closing the program, which is typical of simple demonstration programs but not suitable for a game with a continuous loop. 